\documentclass{ctexart}
\usepackage{text-figure}
\usepackage{choices}
\usepackage{graphicx}
\usepackage[a4paper]{geometry}

\begin{document}



% \textfigure[fig-pos = top-center]{
%   \begin{enumerate}
%     \item 画出 $y = f(x)$ 的图像;
%     \item 当 $x \in [0, +\infty), f(x) < a x + b$, 求 $a + b$ 的最小值.
%   \end{enumerate}
% }{
%   \includegraphics[width = 3cm]{example-image-a}
% }


\textfigure[fig-pos = bottom-right]{
  \begin{enumerate}
    \item 画出 $y = f(x)$ 的图像;
    \item 当 $x \in [0, +\infty), f(x) < a x + b$, 求 $a + b$ 的最小值.
  \end{enumerate}
}{
  \includegraphics[width = 3cm]{example-image-a}
}

\begin{enumerate}
  \item \textfigure[vsep = 0.2\baselineskip, fig-pos = bottom-right]{
      如图, 在平面直角坐标系 $x O y$ 中, 椭圆 $C$ 过点 $\left(\sqrt{3}, \frac{1}{2}\right)$, 焦点为 $F_{1}(-\sqrt{3}, 0), F_{2}(\sqrt{3}, 0)$, 圆 $O$ 的直径为 $F_{1} F_{2}$.
      \begin{enumerate}
        \item 求椭圆 $C$ 及圆 $O$ 的方程;
        \item 设直线 $l$ 与圆 $O$ 相切于第一象限内的点 $P$.
        \begin{enumerate}
            \item 若直线 $l$ 与椭圆 $C$ 有且只有一个公共点, 求点 $P$ 的坐标;
            \item
              直线 $l$ 与椭圆 $C$ 交于 $A, B$ 两点. 若 $\triangle O A B$ 的面积为 $\frac{2 \sqrt{6}}{7}$, 求直 线 $l$ 的方程.   
          \end{enumerate}
      \end{enumerate}
    }{
      \includegraphics[width = 5cm]{example-image-a}
    }
  \item 
  \textfigure[fig-pos = bottom-center]{
    如图, 在正三棱柱 $A B C-A_{1} B_{1} C_{1}$ 中, $A B=A A_{1}=2$, 点 $P, Q$ 分别为 $A_{1} B_{1}, B C$ 的中点.
      \begin{enumerate}
        \item 求异面直线 $B P$ 与 $A C_{1}$ 所成角的余弦值;
        \item 求直线 $C C_{1}$ 与平面 $A Q C_{1}$ 所成角的正弦值.
      \end{enumerate}
    }{
      \includegraphics[width = 5cm]{example-image-a}
    }
\end{enumerate}
% \begin{flushright}
%   \includegraphics[width = 3cm]{example-image-a}
% \end{flushright}
\end{document}