\documentclass{ctexart}
\usepackage{varwidth}
\usepackage{graphicx}
\usepackage{text-figure}

\begin{document}
\ExplSyntaxOn
\coffin_new:N \l_test_coffin

\hcoffin_set:Nn \l_test_coffin
  {
    \begin{varwidth}{\hsize}
      如图, 在正三棱柱 $A B C-A_{1} B_{1} C_{1}$ 中, $A B=A A_{1}=2$, 点 $P, Q$ 分别为 $A_{1} B_{1}, B C$ 的中点.
     如图, 在正三棱柱 $A B C-A_{1} B_{1} C_{1}$ 中, $A B=A A_{1}=2$, 点 $P, Q$ 分别为 $A_{1} B_{1}, B C$ 的中点.
      % 如图, 在正三棱柱 $A B C-A_{1} B_{1} C_{1}$ 中, $A B=A A_{1}=2$, 点 $P, Q$ 分别为 $A_{1} B_{1}, B C$ 的中点.
      \begin{itemize}
        \item 求异面直线 $B P$ 与 $A C_{1}$ 所成角的余弦值;
        \item 求直线 $C C_{1}$ 与平面 $A Q C_{1}$ 所成角的正弦值.
      \end{itemize}
    \end{varwidth}
  }
\hcoffin_set:Nn \l_tmpa_coffin
  {
    \begin{varwidth}{\hsize}
      \includegraphics[width = 2cm]{example-image-a}
    \end{varwidth}
  }
\coffin_join:NnnNnnnn
  \l_test_coffin { b } { l }
  \l_tmpa_coffin { t } { r }
  { 0pt } { 0pt }
\begin{enumerate}
  \item \coffin_typeset:Nnnnn 
    \l_test_coffin { t } { l }
    { 0pt } { 0.5\baselineskip }
  \item \textfigure[]{
    如图, 在正三棱柱 $A B C-A_{1} B_{1} C_{1}$ 中, $A B=A A_{1}=2$, 点 $P, Q$ 分别为 $A_{1} B_{1}, B C$ 的中点.
    如图, 在正三棱柱 $A B C-A_{1} B_{1} C_{1}$ 中, $A B=A A_{1}=2$, 点 $P, Q$ 分别为 $A_{1} B_{1}, B C$ 的中点.
    % 如图, 在正三棱柱 $A B C-A_{1} B_{1} C_{1}$ 中, $A B=A A_{1}=2$, 点 $P, Q$ 分别为 $A_{1} B_{1}, B C$ 的中点.
      \begin{itemize}
        \item 求异面直线 $B P$ 与 $A C_{1}$ 所成角的余弦值;
        \item 求直线 $C C_{1}$ 与平面 $A Q C_{1}$ 所成角的正弦值.
      \end{itemize}
    }{
      \includegraphics[width = 5cm]{example-image-a}
    }
\end{enumerate}


\ExplSyntaxOff
\end{document}