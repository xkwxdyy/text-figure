\documentclass{xdyy-usermanual}

\xdyymanualsetup{
  info = {
    author            = {夏康玮},
    title             = {\pkg{text-figure}宏包},
    email             = {kangweixia_xdyy@163.com},
    date              = {2022-02-01},
    version           = {1.0.2},
    github-repository = {https://github.com/xkwxdyy/},
    gitee-repository  = {https://gitee.com/xkwxdyy/},
  }
}


\usepackage{../xchoices/xchoices}
\usepackage{text-figure}
\usepackage{graphicx}


\begin{document}
\maketitle
\tableofcontents


\section{宏包简介与编写背景}
数学试卷排版中会出现选择题、填空题或是解答题与图片一起排版的情况,
但是图片的对齐往往是比较难处理的一项,考虑到 \cmd{coffin} 具有很好的对齐效果,于是基于 \LaTeX3 的 \cmd{coffin} 模块编写了这个宏包,主要提供了 \tn{textfigure} 命令,使用详情请看 section \ref{sec:命令说明}。


\section{注意事项}

\begin{enumerate}
  \item \tn{textfigure} 命令本身因为要排版文字而编写的,但是本身也可以用输入文本文字等其他内容,可以看成是一个将 \cmd{coffin} 对齐操作包装了的命令;
  \item 本命令并没有图片绕排效果,如果非要绕排,请使用 \pkg{wrapfig} 等绕排宏包,但是由于本 \pkg{text-figure} 宏包的 \tn{textfigure} 命令本身是针对一道题的,所以和绕排宏包并不冲突,完全可以一起使用,即需要绕排用绕排,不需要绕排时使用 \tn{textfigure} 命令排版会方便;
  \item 考虑到用户可能使用的插图命令不同(比如 \tn{includegraphics} 或 \tn{includesvg} 等),且不同的命令可能有不同的可选参数设置,综上考虑没有将插图命令包装进 \tn{textfigure} 命令,也就是用户需要手动输入 \tn{includegraphics} 或 \tn{includesvg} 等命令来插图;
  \item 本命令是基于 \cmd{coffin} 开发,本质是把两个内容放在两个 \cmd{coffin} 盒子里进行对齐,所以不可断页,但根据本人观察试卷排版,绝大部分都是图片和文本在同一页,(也有例外,比如选择题的图片排版,题干和图片不在同一页,这个灵活处理即可,要么就直接插图,要么就用将选择题选项作为 \meta{content} 和图片用 \tn{textfigure} 去对齐
\end{enumerate}

\section{命令说明}\label{sec:命令说明}


\subsection{命令介绍}

\begin{function}[added = 2022-01-22]{ \textfigure }
  \begin{syntax}
    |\textfigure| \oarg{\kvopt{\meta{key}}{\meta{val}}} \marg{content} \marg{figure}
  \end{syntax}
  \oarg{\kvopt{\meta{key}}{\meta{val}}} 指键值设置,\marg{content}指作为主体的文本,\marg{figure}输入插入图片的命令或其他文本。
  \remark{
    \marg{figure} 里可以输入插图命令以外的命令或纯文本文字,因为这个命令是基于前面提到的背景研发的,但是功能要更广。只是要注意在用 \cmd{fig-pos = \meta{postion} } 键值调整对齐方式(下面会详细介绍键值列表)的时候 \meta{position} 是 \meta{content} 不动,\meta{figure} 相对于 \meta{content} 的位置进行设计的(用户可以看下面的示例或者自己尝试使用一下,很快就会有体会)。但是下面为了说明方便,还是用 \meta{content} 指代第一个必选参数的内容,\meta{figure} 指代第二个必选参数的内容。
  }
\end{function}

\begin{function}[added = 2022-01-22]{\textfiguresetup}
  \begin{syntax}
    |\textfiguresetup| = \marg{\kvopt{\meta{key}}{\meta{val}}}
  \end{syntax}
  放在导言区的全局键值设置命令。
\end{function}



\subsection{键值列表}
下面介绍 \tn{textfigure} 命令的键值列表
\begin{function}[added = 2022-01-22, updated = 2022-01-26]{fig-pos}
  \begin{syntax}
    |fig-pos| = \meta{postion} \init{right-top}
  \end{syntax}
  设置 \meta{figure} 相对于 \meta{content}的位置,共有下面几个位置:
    \remark{注意下面所说的“图片”是指 \meta{figure} 的内容,可以不是图片(前文已经解释过),但是为了叙述方便还是说是“图片”,而 \meta{content} 是“文本”}
    \begin{itemize}
      \item top:图片在文本上方
      \item bottom:图片在文本下方
      \item left:图片在文本左边,垂直中心对齐
      \item right:图片在文本右边,垂直中心对齐
      \item top-left:图片在文本的左上方
      \item top-center:图片在文本的上方,位于整行的水平中心
      \item top-right:图片在文本的右上方
      \item top-flushright:图片在文本的右上方,位于水平末端
      \item bottom-left:图片在文本的左下方
      \item bottom-center:图片在文本的下方,位于整行的水平中心
      \item bottom-right:图片在文本的右下方
      \item bottom-flushright:图片在文本的右下方,位于水平末端
      \item left-top:图片在文本左边,顶部对齐
      \item left-center:等价于left
      \item left-bottom:图片在文本左边,底部对齐
      \item right-top:图片在文本右边边,顶部对齐
      \item right-center:等价于right
      \item right-bottom:图片在文本右边,底部对齐
    \end{itemize}
  可能用户注意到了 \meta{position} 里面有几个特别像,比如 \cmd{top-right} 和 \cmd{right-top},它们的区别和联系,以及如果理解这样命名,会在后面的示例中进行说明
\end{function}


\begin{function}[added = 2022-01-22, updated = 2022-01-23]{hsep, vsep}
  \begin{syntax}
    |hsep| = \meta{dimension} \init{0pt}
    |vsep| = \meta{dimension} \init{0pt}
  \end{syntax}
  \cmd{hsep} 和 \cmd{vsep} 分别表示整体的水平和垂直方向的额外偏移量,如果有部分整体的位置不满意可以在命令可选参数中手动调整,如果全都不满意,可以使用 \tn{textfiguresetup} 命令在导言区全局调整。
\end{function}

\begin{function}[added = 2022-01-23]{figure-hsep, figure-vsep}
  \begin{syntax}
    |figure-hsep| = \meta{dimension} \init{0pt}
    |figure-vsep| = \meta{dimension} \init{0pt}
  \end{syntax}
  \cmd{hsep} 和 \cmd{vsep} 分别表示图片的水平和垂直方向的额外偏移量,如果有部分图片的位置不满意可以在命令可选参数中手动调整,如果全都不满意,可以使用 \tn{textfiguresetup} 命令在导言区全局调整。
\end{function}


\begin{function}[added = 2022-01-22]{text-width, figure-width}
  \begin{syntax}
    |text-width| = \meta{dimension} \init{\hsize}
    |figure-width| = \meta{dimension} \init{\hsize}
  \end{syntax}
  文本和图片都置于 \env{varwidth} 环境中,\cmd{text-width} 和 \cmd{figure-width} 分别用来设置文本的 \env{varwidth} 环境宽度和图片的 \env{varwidth} 环境宽度。效果就是如果内容是文本的话就是文本盒子的宽度,如果是用 \tn{includegraphics} 等命令插入的图片,则以 \tn{includegraphics} 设置的宽度为主。
\end{function}


\begin{function}[added = 2022-01-22, updated = 2022-01-26]{text-ratio}
  \begin{syntax}
    |text-ratio| = <fp number>
  \end{syntax}
  调整文本宽度占行宽的比例,取值为 $0-1$。当 \cmd{fig-pos} 为 \cmd{left}, \cmd{right}, \cmd{left-top}, \cmd{left-bottom}, \cmd{right-top}, \cmd{righ-bottom} 时会自动设置为 $0.7$,其余设置为 $1$, 最终以用户手动输入的为主。
  \remark{
    如果同时设置了 \cmd{text-width} 和 \cmd{text-ratio},则以最终效果为 \cmd{text-width} 的设置。
    % 如果是 \cmd{fig-pos = left-*} 或 \cmd{fig-pos = right-*} 和 \cmd{text-ratio} 同时使用,\cmd{text-ratio} 必须放在 \cmd{fig-pos} 设置的后面,否则手动的 \cmd{text-ratio} 的设置会被 \cmd{fig-pos = left-*} 或 \cmd{fig-pos = right-*} 内部的  \cmd{text-ratio} 自动设置覆盖
  }
\end{function}


\section{示例}
\remark{
  注意下面的效果图的尾部会多出一部分是因为示例环境的关系,正常使用是不会的:
}

\fbox{
  \textfigure[fig-pos = right]{
    “好风凭借力,送我上青云。”少年强则中国强,体育强则中国强。希望以北京冬奥会为契机,把冰雪运动各个项目开展起来,让冰上项目强项更强,同时抓紧补上雪地项目短板,把没有开展的项目开展起来,促进冰雪运动在我国全面开展。
  }{
    \includegraphics[width = 2cm]{example-image-a}
  }
}



\subsection{\cmd{right} 系列}


    \textfigure[fig-pos = right]{
      “好风凭借力,送我上青云。”少年强则中国强,体育强则中国强。希望以北京冬奥会为契机,把冰雪运动各个项目开展起来,让冰上项目强项更强,同时抓紧补上雪地项目短板,把没有开展的项目开展起来,促进冰雪运动在我国全面开展。
    }{
      \includegraphics[width = 2cm]{example-image-a}
    }



    \textfigure[fig-pos = right-top]{
      “好风凭借力,送我上青云。”少年强则中国强,体育强则中国强。希望以北京冬奥会为契机,把冰雪运动各个项目开展起来,让冰上项目强项更强,同时抓紧补上雪地项目短板,把没有开展的项目开展起来,促进冰雪运动在我国全面开展。
    }{
      \includegraphics[width = 2cm]{example-image-a}
    }



    \textfigure[fig-pos = right-bottom]{
      “好风凭借力,送我上青云。”少年强则中国强,体育强则中国强。希望以北京冬奥会为契机,把冰雪运动各个项目开展起来,让冰上项目强项更强,同时抓紧补上雪地项目短板,把没有开展的项目开展起来,促进冰雪运动在我国全面开展。
    }{
      \includegraphics[width = 2cm]{example-image-a}
    }




\subsection{\cmd{left} 系列}


    \textfigure[fig-pos = left]{
      “好风凭借力,送我上青云。”少年强则中国强,体育强则中国强。希望以北京冬奥会为契机,把冰雪运动各个项目开展起来,让冰上项目强项更强,同时抓紧补上雪地项目短板,把没有开展的项目开展起来,促进冰雪运动在我国全面开展。
    }{
      \includegraphics[width = 2cm]{example-image-a}
    }



    \textfigure[fig-pos = left-top]{
      “好风凭借力,送我上青云。”少年强则中国强,体育强则中国强。希望以北京冬奥会为契机,把冰雪运动各个项目开展起来,让冰上项目强项更强,同时抓紧补上雪地项目短板,把没有开展的项目开展起来,促进冰雪运动在我国全面开展。
    }{
      \includegraphics[width = 2cm]{example-image-a}
    }



    \textfigure[fig-pos = left-bottom]{
      “好风凭借力,送我上青云。”少年强则中国强,体育强则中国强。希望以北京冬奥会为契机,把冰雪运动各个项目开展起来,让冰上项目强项更强,同时抓紧补上雪地项目短板,把没有开展的项目开展起来,促进冰雪运动在我国全面开展。
    }{
      \includegraphics[width = 2cm]{example-image-a}
    }




\subsection{\cmd{top} 系列}

\subsubsection{\cmd{top} 的三位方位}


  \textfigure[fig-pos = top]{
    “好风凭借力,送我上青云。”少年强则中国强,体育强则中国强。希望以北京冬奥会为契机,把冰雪运动各个项目开展起来,让冰上项目强项更强,同时抓紧补上雪地项目短板,把没有开展的项目开展起来,促进冰雪运动在我国全面开展。
  }{
    \includegraphics[width = 2cm]{example-image-a}
  }




    \textfigure[fig-pos = top-left]{
      “好风凭借力,送我上青云。”少年强则中国强,体育强则中国强。希望以北京冬奥会为契机,把冰雪运动各个项目开展起来,让冰上项目强项更强,同时抓紧补上雪地项目短板,把没有开展的项目开展起来,促进冰雪运动在我国全面开展。
    }{
      \includegraphics[width = 2cm]{example-image-a}
    }



    \textfigure[fig-pos = top-center]{
      “好风凭借力,送我上青云。”少年强则中国强,体育强则中国强。希望以北京冬奥会为契机,把冰雪运动各个项目开展起来,让冰上项目强项更强,同时抓紧补上雪地项目短板,把没有开展的项目开展起来,促进冰雪运动在我国全面开展。
    }{
      \includegraphics[width = 2cm]{example-image-a}
    }



    \textfigure[fig-pos = top-right]{
      “好风凭借力,送我上青云。”少年强则中国强,体育强则中国强。希望以北京冬奥会为契机,把冰雪运动各个项目开展起来,让冰上项目强项更强,同时抓紧补上雪地项目短板,把没有开展的项目开展起来,促进冰雪运动在我国全面开展。
    }{
      \includegraphics[width = 2cm]{example-image-a}
    }



\subsubsection{\cmd{top} 和 \cmd{top-center} 的区别}


  \textfigure[fig-pos = top]{
    \begin{enumerate}
      \item 画出 $y = f(x)$ 的图像;
      \item 当 $x \in [0, +\infty), f(x) < a x + b$, 求 $a + b$ 的最小值.
    \end{enumerate}
  }{
    \includegraphics[width = 3cm]{example-image-a}
  }



    \textfigure[fig-pos = top-center]{
      \begin{enumerate}
        \item 画出 $y = f(x)$ 的图像;
        \item 当 $x \in [0, +\infty), f(x) < a x + b$, 求 $a + b$ 的最小值.
      \end{enumerate}
    }{
      \includegraphics[width = 3cm]{example-image-a}
    }



\subsubsection{\cmd{top-right} 和 \cmd{top-flushright} 的区别}


  \textfigure[fig-pos = top-right]{
    \begin{enumerate}
      \item 画出 $y = f(x)$ 的图像;
      \item 当 $x \in [0, +\infty), f(x) < a x + b$, 求 $a + b$ 的最小值.
    \end{enumerate}
  }{
    \includegraphics[width = 3cm]{example-image-a}
  }



  \textfigure[fig-pos = top-flushright]{
    \begin{enumerate}
      \item 画出 $y = f(x)$ 的图像;
      \item 当 $x \in [0, +\infty), f(x) < a x + b$, 求 $a + b$ 的最小值.
    \end{enumerate}
  }{
    \includegraphics[width = 3cm]{example-image-a}
  }


\subsection{\cmd{bottom} 系列}

\subsubsection{\cmd{bottom} 的三位方位}


  \textfigure[fig-pos = bottom-left]{
    “好风凭借力,送我上青云。”少年强则中国强,体育强则中国强。希望以北京冬奥会为契机,把冰雪运动各个项目开展起来,让冰上项目强项更强,同时抓紧补上雪地项目短板,把没有开展的项目开展起来,促进冰雪运动在我国全面开展。
  }{
    \includegraphics[width = 2cm]{example-image-a}
  }



    \textfigure[fig-pos = bottom-center]{
      “好风凭借力,送我上青云。”少年强则中国强,体育强则中国强。希望以北京冬奥会为契机,把冰雪运动各个项目开展起来,让冰上项目强项更强,同时抓紧补上雪地项目短板,把没有开展的项目开展起来,促进冰雪运动在我国全面开展。
    }{
      \includegraphics[width = 2cm]{example-image-a}
    }



  \textfigure[fig-pos = bottom-right]{
    “好风凭借力,送我上青云。”少年强则中国强,体育强则中国强。希望以北京冬奥会为契机,把冰雪运动各个项目开展起来,让冰上项目强项更强,同时抓紧补上雪地项目短板,把没有开展的项目开展起来,促进冰雪运动在我国全面开展。
  }{
    \includegraphics[width = 3cm]{example-image-a}
  }


\subsubsection{\cmd{bottom-right} 和 \cmd{bottom-flushright} 的区别}


  \textfigure[fig-pos = bottom-right]{
    \begin{enumerate}
      \item 画出 $y = f(x)$ 的图像;
      \item 当 $x \in [0, +\infty), f(x) < a x + b$, 求 $a + b$ 的最小值.
    \end{enumerate}
  }{
    \includegraphics[width = 3cm]{example-image-a}
  }



  \textfigure[fig-pos = bottom-flushright]{
    \begin{enumerate}
      \item 画出 $y = f(x)$ 的图像;
      \item 当 $x \in [0, +\infty), f(x) < a x + b$, 求 $a + b$ 的最小值.
    \end{enumerate}
  }{
    \includegraphics[width = 3cm]{example-image-a}
  }


\subsubsection{\cmd{bottom} 和 \cmd{bottom-center} 的区别}


  \textfigure[fig-pos = bottom]{
    \begin{enumerate}
      \item 画出 $y = f(x)$ 的图像;
      \item 当 $x \in [0, +\infty), f(x) < a x + b$, 求 $a + b$ 的最小值.
    \end{enumerate}
  }{
    \includegraphics[width = 3cm]{example-image-a}
  }



    \textfigure[fig-pos = bottom-center]{
      \begin{enumerate}
        \item 画出 $y = f(x)$ 的图像;
        \item 当 $x \in [0, +\infty), f(x) < a x + b$, 求 $a + b$ 的最小值.
      \end{enumerate}
    }{
      \includegraphics[width = 3cm]{example-image-a}
    }






\subsection{ \cmd{fig-pos} 总结}

从上面的 \cmd{fig-pos} 示例可以看出,其实很容易记住这些方位,拿 \meta{pos1}-\meta{pos2}为例:
  \begin{itemize}
    \item \meta{pos1} 其实就是图片整体在文本的位置,就是上下左右四个位置
    \item \meta{pos2} 是位置的细节,比如 \meta{pos1} 为 \cmd{top}, \cmd{bottom} 的时候 \meta{pos2} 就是左右( \cmd{left}, \cmd{right} ),\meta{pos1} 为 \cmd{left}, \cmd{right} 的时候 \meta{pos2} 就是上中下(或顶部中部底部),即 \cmd{top}, \cmd{center}, \cmd{bottom}, 特别地 \cmd{center} 可以省略,即 \cmd{right} 为 \cmd{right-center} 的效果
    \item 关于  \cmd{top} 和 \cmd{bottom} 系列,根据需要增加了两个键值: \cmd{top-flushright} 和 \cmd{bottom-flushright} ,实现了图片水平方向上有 \env{flushright} 环境的效果
    \item 为了满足需求和考虑到命令的应用广泛性,将 \cmd{top} 和 \cmd{top-center} 分别设置为两种效果,前者是文本正上方,后者是文本上方,处于行中心( \cmd{bottom} 与 \cmd{bottom-center} 同理)
  \end{itemize}

\remark{
  \cmd{fig-pos} 的设置已经照顾到绝大情形的应用需求,由于实现机制的原因,可能会造成图片视觉上有一小点的偏移(比如文本某些换行机制导致可能会超出该行一小段距离,或是文本较短,图片水平方向要超出一点(比如 \cmd{fig-pos = bottom-flushright} 的图片是具有 \env{flushright} 的效果的,可能就造成超出文字的效果)),精益求精的用户可以用 \cmd{figure-sep} 进行方便的调节
}

\subsection{ \cmd{text-ratio} 和 \cmd{text-width} }



\subsubsection{ \cmd{text-ratio} 的作用 }


    \textfigure[fig-pos = right, text-ratio = 0.3]{
      “好风凭借力,送我上青云。”少年强则中国强,体育强则中国强。希望以北京冬奥会为契机,把冰雪运动各个项目开展起来,让冰上项目强项更强,同时抓紧补上雪地项目短板,把没有开展的项目开展起来,促进冰雪运动在我国全面开展。
    }{
      \includegraphics[width = 2cm]{example-image-a}
    }



    \textfigure[fig-pos = right, text-ratio = 0.5]{
      “好风凭借力,送我上青云。”少年强则中国强,体育强则中国强。希望以北京冬奥会为契机,把冰雪运动各个项目开展起来,让冰上项目强项更强,同时抓紧补上雪地项目短板,把没有开展的项目开展起来,促进冰雪运动在我国全面开展。
    }{
      \includegraphics[width = 2cm]{example-image-a}
    }



    \textfigure[fig-pos = right, text-ratio = 0.7]{
      “好风凭借力,送我上青云。”少年强则中国强,体育强则中国强。希望以北京冬奥会为契机,把冰雪运动各个项目开展起来,让冰上项目强项更强,同时抓紧补上雪地项目短板,把没有开展的项目开展起来,促进冰雪运动在我国全面开展。
    }{
      \includegraphics[width = 2cm]{example-image-a}
    }



    \textfigure[fig-pos = right, text-ratio = 0.5, text-width = 8cm]{
      “好风凭借力,送我上青云。”少年强则中国强,体育强则中国强。希望以北京冬奥会为契机,把冰雪运动各个项目开展起来,让冰上项目强项更强,同时抓紧补上雪地项目短板,把没有开展的项目开展起来,促进冰雪运动在我国全面开展。
    }{
      \includegraphics[width = 2cm]{example-image-a}
    }



\subsubsection{ \cmd{text-width} 的作用 }


    \textfigure[fig-pos = right, text-width = 6cm]{
      “好风凭借力,送我上青云。”少年强则中国强,体育强则中国强。希望以北京冬奥会为契机,把冰雪运动各个项目开展起来,让冰上项目强项更强,同时抓紧补上雪地项目短板,把没有开展的项目开展起来,促进冰雪运动在我国全面开展。
    }{
      \includegraphics[width = 2cm]{example-image-a}
    }



    \textfigure[fig-pos = right, text-width = 10cm]{
      “好风凭借力,送我上青云。”少年强则中国强,体育强则中国强。希望以北京冬奥会为契机,把冰雪运动各个项目开展起来,让冰上项目强项更强,同时抓紧补上雪地项目短板,把没有开展的项目开展起来,促进冰雪运动在我国全面开展。
    }{
      \includegraphics[width = 2cm]{example-image-a}
    }




\subsubsection{ \cmd{text-ratio} 和 \cmd{text-width} 同时使用}


    \textfigure[fig-pos = right, text-ratio = 0.5]{
      “好风凭借力,送我上青云。”少年强则中国强,体育强则中国强。希望以北京冬奥会为契机,把冰雪运动各个项目开展起来,让冰上项目强项更强,同时抓紧补上雪地项目短板,把没有开展的项目开展起来,促进冰雪运动在我国全面开展。
    }{
      \includegraphics[width = 2cm]{example-image-a}
    }



    \textfigure[fig-pos = right, text-ratio = 0.5, text-width = 8cm]{
      “好风凭借力,送我上青云。”少年强则中国强,体育强则中国强。希望以北京冬奥会为契机,把冰雪运动各个项目开展起来,让冰上项目强项更强,同时抓紧补上雪地项目短板,把没有开展的项目开展起来,促进冰雪运动在我国全面开展。
    }{
      \includegraphics[width = 2cm]{example-image-a}
    }




\section{试卷排版示例}


\subsection{选择题}

\begin{enumerate}
  \item \textfigure[fig-pos = bottom]{
    如图, 网格纸上小正方形的边长为 $1$ , 粗实线画出的是某几何体的三视图, 该几何体由一平面将一圆柱截去一部分后所得, 则该几何体的体积为 ( )
  }{
    \includegraphics[width = 3cm]{example-image-a}
  }
    \begin{xchoices}
      \item $90\pi$
      \item $63\pi$
      \item $42\pi$
      \item $36\pi$ 
    \end{xchoices}
\end{enumerate}


\subsection{填空题}

\begin{enumerate}[start = 11]
  \item \textfigure[fig-pos = right-top]{
    执行如图的程序框图, 如果输入的 $a = -1$, 则输出的 $S = \underline{\qquad}$
  }{
    \includegraphics[height = 3cm]{example-image-a}
  }
\end{enumerate}



\subsection{解答题}

\begin{enumerate}[start = 18]
  \item \textfigure[fig-pos = bottom-right]{
    如图, 在正三棱柱 $A B C-A_{1} B_{1} C_{1}$ 中, $A B=A A_{1}=2$, 点 $P, Q$ 分别为 $A_{1} B_{1}, B C$ 的中点.
      \begin{enumerate}
        \item 求异面直线 $B P$ 与 $A C_{1}$ 所成角的余弦值;
        \item 求直线 $C C_{1}$ 与平面 $A Q C_{1}$ 所成角的正弦值.
      \end{enumerate}
    }{
      \includegraphics[width = 5cm]{example-image-a}
    }
  \item \textfigure[vsep = 0.2\baselineskip, fig-pos = bottom-right]{
  % 如图, 在正三棱柱 $A B C-A_{1} B_{1} C_{1}$ 中, $A B=A A_{1}=2$, 点 $P, Q$ 分别为 $A_{1} B_{1}, B C$ 的中点.
    如图, 在平面直角坐标系 $x O y$ 中, 椭圆 $C$ 过点 $\left(\sqrt{3}, \frac{1}{2}\right)$, 焦点为 $F_{1}(-\sqrt{3}, 0), F_{2}(\sqrt{3}, 0)$, 圆 $O$ 的直径为 $F_{1} F_{2}$.
    \begin{enumerate}
      \item 求椭圆 $C$ 及圆 $O$ 的方程;
      \item 设直线 $l$ 与圆 $O$ 相切于第一象限内的点 $P$.
      \begin{enumerate}
          \item 若直线 $l$ 与椭圆 $C$ 有且只有一个公共点, 求点 $P$ 的坐标;
          \item
            直线 $l$ 与椭圆 $C$ 交于 $A, B$ 两点. 若 $\triangle O A B$ 的面积为 $\frac{2 \sqrt{6}}{7}$, 求直 线 $l$ 的方程.   
        \end{enumerate}
    \end{enumerate}
  }{
    \includegraphics[width = 5cm]{example-image-a}
  }
\end{enumerate}

\end{document}
