\documentclass{ctexart}
\usepackage{text-pic}
\usepackage{graphicx}
\usepackage{choices}

\begin{document}

\begin{enumerate}
  \item \textpic[anchor = east]{
    某工厂为提高生产效率,开展技术创新活动,提出了完成某项生产任务的两种新的生产方式.为比较两种生产方式的效率,选取$40$名工人,将他们随机分成两组,每组$20$人.第一组工人用第一种生产方式,第二组工人用第二种生产方式.根据工人完成生产任务的工作时间绘制了如下茎叶图:
    \choices{
      1 &&
      2 &&
      3 &&
      4 &&
    }
  }{
    \includegraphics[width = 2cm]{example-image-a}
  }
  
  \item 某工厂为提高生产效率,开展技术创新活动,提出了完成某项生产任务的两种新的生产方式.为比较两种生产方式的效率,选取$40$名工人,将他们随机分成两组,每组$20$人.第一组工人用第一种生产方式,第二组工人用第二种生产方式.根据工人完成生产任务的工作时间绘制了如下茎叶图:
  
  \textpic[anchor = north]{
    \choices{
      1 &&
      2 &&
      3 &&
      4 &&
    }
  }{
    \includegraphics[width = 2cm]{example-image-a}
  }
\end{enumerate}

\begin{enumerate}
  \item \textpic[anchor = southeast]{
    某工厂为提高生产效率,开展技术创新活动,提出了完成某项生产任务的两种新的生产方式.为比较两种生产方式的效率,选取$40$名工人,将他们随机分成两组,每组$20$人.第一组工人用第一种生产方式,第二组工人用第二种生产方式.根据工人完成生产任务的工作时间绘制了如下茎叶图:
    }{
      \includegraphics[width = 4cm]{example-image-a}
    }
  \item \textpic[anchor = southeast]{
    某工厂为提高生产效率,开展技术创新活动,提出了完成某项生产任务的两种新的生产方式.为比较两种生产方式的效率,选取$40$名工人,将他们随机分成两组,每组$20$人.第一组工人用第一种生产方式,第二组工人用第二种生产方式.根据工人完成生产任务的工作时间绘制了如下茎叶图:
    }{
      \includegraphics[width = 4cm]{example-image-a}
    }
\end{enumerate}
\end{document}