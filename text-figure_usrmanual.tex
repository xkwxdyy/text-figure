% 由于使用了minted宏包排版代码,
% 需要安装Python
% 并命令行使用pip install Pygments安装Pygments库
% 然后编译需要使用xelatex --shell-escape选项
\documentclass{l3doc}
\usepackage{ctex}
\usepackage{hyperref}
\usepackage{xcolor}
\usepackage{listings}
\usepackage{varwidth}      % 获得宽度
\usepackage{graphicx}
\title{\bfseries text-figure宏包}
\author{夏康玮\\ \path{kangweixia_xdyy@163.com}}
\date{2022-01-22\quad v1.0.0 \thanks{\url{https://github.com/xkwxdyy/text-figure}} \thanks{\url{https://gitee.com/xkwxdyy/text-figure}}}

\usepackage{minted}
\setminted{formatcom=\xeCJKVerbAddon}
\usepackage{sourcecodepro}
\usepackage{xcolor}
\usepackage{text-figure}
\usepackage{choices}
\usepackage{mwe}
\makeatletter
\newlength\savefboxrule
\newlength\savefboxsep
\edef\example@name{\jobname-example.aux}
\newenvironment{hexample}%
{%
\renewcommand{\minted@FVB@VerbatimOut}[1]{%
  \@bsphack
  \begingroup
    \FV@DefineWhiteSpace
    \def\FV@Space{\space}%
    \FV@DefineTabOut
    \let\FV@ProcessLine\minted@write@detok
    \immediate\openout\FV@OutFile ##1\relax
    \let\FV@FontScanPrep\relax
    \let\@noligs\relax
    \FV@Scan}
\minted@FVB@VerbatimOut{\example@name}}%
{\minted@FVE@VerbatimOut%
  \trivlist\item\relax
  \setlength{\savefboxrule}{\fboxrule}%
  \setlength{\savefboxsep}{\fboxsep}%
  \setlength{\fboxsep}{0.015\textwidth}%
  \setlength{\fboxrule}{0.4pt}%
  \fcolorbox[gray]{0}{0.95}{%
    \begin{minipage}[c]{0.45\textwidth}%
      \setlength{\fboxrule}{\savefboxrule}%
      \setlength{\fboxsep}{\savefboxsep}%
      \small\inputminted[breakanywhere,breaklines=true,numbers=left]{latex}{\example@name}%
    \end{minipage}%
  }%
  \hfill%
  \fbox{%
    \begin{minipage}[c]{0.45\textwidth}%
      \setlength{\fboxrule}{\savefboxrule}%
      \setlength{\fboxsep}{\savefboxsep}%
      \setlength{\parskip}{1ex plus 0.4ex minus 0.2ex}%
      \normalsize\input{\example@name}%
    \end{minipage}%
  }%
  \endtrivlist
}
\newenvironment{vexample}%
{%
\renewcommand{\minted@FVB@VerbatimOut}[1]{%
  \@bsphack
  \begingroup
    \FV@DefineWhiteSpace
    \def\FV@Space{\space}%
    \FV@DefineTabOut
    \let\FV@ProcessLine\minted@write@detok
    \immediate\openout\FV@OutFile ##1\relax
    \let\FV@FontScanPrep\relax
    \let\@noligs\relax
    \FV@Scan}
\minted@FVB@VerbatimOut{\example@name}}%
{\minted@FVE@VerbatimOut%
  \trivlist\item\relax
  \setlength{\savefboxrule}{\fboxrule}%
  \setlength{\savefboxsep}{\fboxsep}%
  \setlength{\fboxsep}{0.015\textwidth}%
  \setlength{\fboxrule}{0.4pt}%
  \begin{minipage}[c]{\textwidth}
  \fcolorbox[gray]{0}{0.95}{%
    \begin{minipage}[c]{\textwidth-0.03\textwidth-0.8pt}%
      \setlength{\fboxrule}{\savefboxrule}%
      \setlength{\fboxsep}{\savefboxsep}%
      \small\inputminted[breakanywhere,breaklines=true,numbers=left]{latex}{\example@name}%
    \end{minipage}%
  }\\
  \fbox{%
    \begin{minipage}[c]{\textwidth-0.03\textwidth-0.8pt}%
    % \begin{varwidth}{\hsize}%
      \setlength{\fboxrule}{\savefboxrule}%
      \setlength{\fboxsep}{\savefboxsep}%
      \setlength{\parskip}{1ex plus 0.4ex minus 0.2ex}%
      \normalsize\input{\example@name}%
    \end{minipage}%
    % \end{varwidth}%
  }%
  \end{minipage}
  \endtrivlist
}
\newenvironment{floatexample}%
{%
\renewcommand{\minted@FVB@VerbatimOut}[1]{%
  \@bsphack
  \begingroup
    \FV@DefineWhiteSpace
    \def\FV@Space{\space}%
    \FV@DefineTabOut
    \let\FV@ProcessLine\minted@write@detok
    \immediate\openout\FV@OutFile ##1\relax
    \let\FV@FontScanPrep\relax
    \let\@noligs\relax
    \FV@Scan}
\minted@FVB@VerbatimOut{\example@name}}%
{\minted@FVE@VerbatimOut%
  \trivlist\item\relax
  \setlength{\savefboxrule}{\fboxrule}%
  \setlength{\savefboxsep}{\fboxsep}%
  \setlength{\fboxsep}{0.015\textwidth}%
  \setlength{\fboxrule}{0.4pt}%
  \fcolorbox[gray]{0}{0.95}{%
    \begin{minipage}[c]{\textwidth-0.03\textwidth-0.8pt}%
      \setlength{\fboxrule}{\savefboxrule}%
      \setlength{\fboxsep}{\savefboxsep}%
      \small\inputminted[breakanywhere,breaklines=true,numbers=left]{latex}{\example@name}%
    \end{minipage}%
  }\par
  \input{\example@name}%
  \endtrivlist
}
\makeatother

\lstnewenvironment{LaTeXdemo}[1][0.4]{
  \lstset{
    basicstyle         = \ttfamily\small,
    basewidth          = 0.51em,
    backgroundcolor    = \color{cyan!10},
    % rulecolor          = \color{red!10},
    frame              = shadowbox,
    frameround         = tttt,
    % framerule          = 0pt,
    gobble             = 2,
    tabsize            = 2,
    numbers            = left,
    language           = [LaTeX]TeX,
    linewidth          = #1\linewidth
  }
}{}
\ExplSyntaxOn
\dim_new:N \g__LaTeXautodemo_width_dim
\box_new:N \g__LaTeXautodemo_width_measure_box
\makeatletter
\lst@RequireAspects{writefile}
\newsavebox\LaTeXautodemo@box
\lstnewenvironment{LaTeXautodemo}[1][code and example]
  {%
    \global\let\lst@intname\@empty
    \edef\LaTeXautodemo@end{%
      \expandafter\noexpand\csname LaTeXautodemo@@#1@end\endcsname
    }%
    \@nameuse{LaTeXautodemo@@#1}%
  }
  {
    \LaTeXautodemo@end
  }
\newcommand\LaTeXautodemo@new[3]{%
  \@namedef{LaTeXautodemo@@#1}{#2}%
  \@namedef{LaTeXautodemo@@#1@end}{#3}%
}
\newcommand*\LaTeXautodemo@common{%
  \setkeys{lst}
    {%
      basicstyle         = \ttfamily\small,
      basewidth          = 0.51em,
      backgroundcolor    = \color{cyan!10},
      % rulecolor          = \color{red!10},
      frame              = shadowbox,
      frameround         = tttt,
      % framerule          = 0pt,
      gobble             = 2,
      tabsize            = 2,
      numbers            = left,
      language           = [LaTeX]TeX,
      % linewidth          = 0.7\linewidth
    }%
}
\newcount\LaTeXautodemo@count
\newcommand*\LaTeXautodemo@input{%
  \catcode`\^^M = 10\relax
  \input{\jobname-\number\LaTeXautodemo@count.tmp}%
}
\LaTeXautodemo@new{code and example}{%
  \setbox\LaTeXautodemo@box=\hbox\bgroup
    \global\advance\LaTeXautodemo@count by 1 %
    \lst@BeginAlsoWriteFile{\jobname-\number\LaTeXautodemo@count.tmp}   % 将内容存在中途文件里面
    \LaTeXautodemo@common
}{%
    \lst@EndWriteFile
  \egroup
  \hbox_set:Nn \g__LaTeXautodemo_width_measure_box
    {
      \begin{varwidth}{\hsize}
        \usebox\LaTeXautodemo@box
      \end{varwidth}
    }
  \dim_set:Nn \g__LaTeXautodemo_width_dim { \box_wd:N \g__LaTeXautodemo_width_measure_box }
  % \lstset
  % 	{
  % 		linewidth = \dim_use:N \l__LaTeXautodemo_width_dim
  % 	}
  \begin{center}
    \dim_compare:nNnTF { \g__LaTeXautodemo_width_dim } > { 0.4\linewidth}
      {
        % \dim_use:N \g__LaTeXautodemo_width_dim
        \begin{minipage}{\linewidth}
          % \setkeys{lst}{ linewidth = \dim_use:N \l__LaTeXautodemo_width_dim }
          \usebox\LaTeXautodemo@box
        \end{minipage}%
        \par
        \begin{minipage}{\linewidth}
          \LaTeXautodemo@input
        \end{minipage}
      }
      {
        \begin{minipage}{0.4\linewidth}
          \usebox\LaTeXautodemo@box
        \end{minipage}%
        \hfil
        \begin{minipage}{0.4\linewidth}
          \LaTeXautodemo@input
        \end{minipage}%
      }
    % \ifdim\wd\LaTeXautodemo@box > 0.48\linewidth
    %   \begin{minipage}{\linewidth}
    %     \usebox\LaTeXautodemo@box
    %   \end{minipage}%
    %   \par
    %   \begin{minipage}{\linewidth}
    %     \LaTeXautodemo@input
    %   \end{minipage}
    % \else
    %   \begin{minipage}{0.48\linewidth}
    %     \LaTeXautodemo@input
    %   \end{minipage}%
    %   \hfil
    %   \begin{minipage}{0.48\linewidth}
    %     \usebox\LaTeXautodemo@box
    %   \end{minipage}%
    % \fi
  \end{center}
}
\LaTeXautodemo@new{code and float}{%
  \global\advance\LaTeXautodemo@count by 1 %
  \lst@BeginAlsoWriteFile{\jobname-\number\LaTeXautodemo@count.tmp}%
  \LaTeXautodemo@common
}{%
  \lst@EndWriteFile
  \LaTeXautodemo@input
}
\LaTeXautodemo@new{code only}{\LaTeXautodemo@common}{}
\makeatother
\ExplSyntaxOff

\renewcommand{\emph}[1]{\begingroup \bfseries \textcolor{red!80}{#1} \endgroup}
\NewDocumentCommand{\init}{+v}{\hspace{\fill}初始值~=~\textcolor{blue}{\bfseries#1}}
\DeclareDocumentCommand\kvopt{mm}
  {\texttt{#1 \breakablethinspace = \breakablethinspace#2}}
\def\breakablethinspace{\hskip 0.16667em\relax}
\ExplSyntaxOn
\NewDocumentCommand{ \remark }{ +m }{
  \group_begin:
    \par \large
    \color{violet}
    作者注:
      #1
  \group_end:
}
\ExplSyntaxOff

\begin{document}
\maketitle
\tableofcontents


\section{宏包简介与编写背景}
数学试卷排版中会出现选择题、填空题或是解答题与图片一起排版的情况,
但是图片的对齐往往是比较难处理的一项,考虑到 \cmd{coffin} 具有很好的对齐效果,于是基于 \LaTeX3 的 \cmd{coffin} 模块编写了这个宏包,主要提供了 \tn{textfigure} 命令,使用详情请看 section \ref{sec:命令说明}。


\section{注意事项}

\begin{enumerate}
  \item \tn{textfigure} 命令本身因为要排版文字而编写的,但是本身也可以用输入文本文字等其他内容,可以看成是一个将 \cmd{coffin} 对齐操作包装了的命令;
  \item 本命令并没有图片绕排效果,如果非要绕排,请使用 \pkg{wrapfig} 等绕排宏包,但是由于本 \pkg{text-figure} 宏包的 \tn{textfigure} 命令本身是针对一道题的,所以和绕排宏包并不冲突,完全可以一起使用,即需要绕排用绕排,不需要绕排时使用 \tn{textfigure} 命令排版会方便;
  \item 考虑到用户可能使用的插图命令不同(比如 \tn{includegraphics} 或 \tn{includesvg} 等),且不同的命令可能有不同的可选参数设置,综上考虑没有将插图命令包装进 \tn{textfigure} 命令,也就是用户需要手动输入 \tn{includegraphics} 或 \tn{includesvg} 等命令来插图;
\end{enumerate}

\section{命令说明}\label{sec:命令说明}


\subsection{命令介绍}

\begin{function}[added = 2022-01-22]{ \textfigure }
  \begin{syntax}
    |\textfigure| \oarg{\kvopt{\meta{key}}{\meta{val}}} \marg{content} \marg{figure}
  \end{syntax}
  \oarg{\kvopt{\meta{key}}{\meta{val}}} 指键值设置,\marg{content}指作为主体的文本,\marg{figure}输入插入图片的命令或其他文本。
  \remark{
    \marg{figure} 里可以输入插图命令以外的命令或纯文本文字,因为这个命令是基于前面提到的背景研发的,但是功能要更广。只是要注意在用 \cmd{fig-pos = \meta{postion} } 键值调整对齐方式(下面会详细介绍键值列表)的时候 \meta{position} 是 \meta{content} 不动,\meta{figure} 相对于 \meta{content} 的位置进行设计的(用户可以看下面的示例或者自己尝试使用一下,很快就会有体会)。但是下面为了说明方便,还是用 \meta{content} 指代第一个必选参数的内容,\meta{figure} 指代第二个必选参数的内容。
  }
\end{function}

\begin{function}[added = 2022-01-22]{\textfiguresetup}
  \begin{syntax}
    |\textfiguresetup| = \marg{\kvopt{\meta{key}}{\meta{val}}}
  \end{syntax}
  放在导言区的全局键值设置命令。
\end{function}



\subsection{键值列表}
下面介绍 \tn{textfigure} 命令的键值列表
\begin{function}[added = 2022-01-22, updated = 2022-01-22]{fig-pos}
  \begin{syntax}
    |fig-pos| = \meta{postion} \init{bottom-right}
  \end{syntax}
  设置 \meta{figure} 相对于 \meta{content}的位置,共有下面几个位置:
    \remark{注意下面所说的“图片”是指 \meta{figure} 的内容,可以不是图片(前文已经解释过),但是为了叙述方便还是说是“图片”,而 \meta{content} 是“文本”}
    \begin{itemize}
      \item top:图片在文本上方
      \item bottom:图片在文本下方
      \item left:图片在文本左边,垂直中心对齐
      \item right:图片在文本右边,垂直中心对齐
      \item top-left:图片在文本的左上方
      \item top-right:图片在文本的右上方
      \item bottom-left:图片在文本的左下方
      \item bottom-right:图片在文本的右下方
      \item left-top:图片在文本左边,顶部对齐
      \item left-center:等价于left
      \item left-bottom:图片在文本左边,底部对齐
      \item right-top:图片在文本右边边,顶部对齐
      \item right-center:等价于right
      \item right-bottom:图片在文本右边,底部对齐
    \end{itemize}
  可能用户注意到了 \meta{position} 里面有几个特别像,比如 \cmd{top-right} 和 \cmd{right-top},它们的区别和联系,以及如果理解这样命名,会在后面的示例中进行说明
\end{function}


\begin{function}[added = 2022-01-22]{hsep, vsep}
  \begin{syntax}
    |hsep| = \meta{dimension} \init{0pt}
    |vsep| = \meta{dimension} \init{0pt}
  \end{syntax}
  \cmd{hsep} 和 \cmd{vsep} 分别表示图片的水平和垂直方向的额外偏移量,如果有部分图片的位置不满意可以在命令可选参数中手动调整,如果全都不满意,可以使用 \tn{textfiguresetup} 命令在导言区全局调整。
\end{function}


\begin{function}[added = 2022-01-22]{text-width, figure-width}
  \begin{syntax}
    |text-width| = \meta{dimension} \init{\hsize}
    |figure-width| = \meta{dimension} \init{\hsize}
  \end{syntax}
  文本和图片都置于 \env{varwidth} 环境中,\cmd{text-width} 和 \cmd{figure-width} 分别用来设置文本的 \env{varwidth} 环境宽度和图片的 \env{varwidth} 环境宽度。效果就是如果内容是文本的话就是文本盒子的宽度,如果是用 \tn{includegraphics} 等命令插入的图片,则以 \tn{includegraphics} 设置的宽度为主。
\end{function}


\begin{function}[added = 2022-01-22]{text-ratio}
  \begin{syntax}
    |text-ratio| = <fp number>
  \end{syntax}
  调整文本宽度占行宽的比例,取值为 $0-1$。当 \cmd{fig-pos} 为 \cmd{left}, \cmd{right}, \cmd{left-top}, \cmd{left-bottom}, \cmd{right-top}, \cmd{righ-bottom} 时会自动设置为 $0.7$,其余设置为 $1$, 最终以用户手动输入的为主。
  \remark{
    如果同时设置了 \cmd{text-width} 和 \cmd{text-ratio},则以最终效果为 \cmd{text-width} 的设置。
  }
\end{function}


\section{示例}
\remark{
  注意下面的效果图的尾部会多出一部分是因为示例环境的关系,正常使用是不会的:
}

\fbox{
  \textfigure[fig-pos = right]{
    “好风凭借力,送我上青云。”少年强则中国强,体育强则中国强。希望以北京冬奥会为契机,把冰雪运动各个项目开展起来,让冰上项目强项更强,同时抓紧补上雪地项目短板,把没有开展的项目开展起来,促进冰雪运动在我国全面开展。
  }{
    \includegraphics[width = 2cm]{example-image-a}
  }
}



\subsection{\cmd{right} 系列}

\begin{vexample}
    \textfigure[fig-pos = right]{
      “好风凭借力,送我上青云。”少年强则中国强,体育强则中国强。希望以北京冬奥会为契机,把冰雪运动各个项目开展起来,让冰上项目强项更强,同时抓紧补上雪地项目短板,把没有开展的项目开展起来,促进冰雪运动在我国全面开展。
    }{
      \includegraphics[width = 2cm]{example-image-a}
    }
\end{vexample}

\begin{vexample}
    \textfigure[fig-pos = right-top]{
      “好风凭借力,送我上青云。”少年强则中国强,体育强则中国强。希望以北京冬奥会为契机,把冰雪运动各个项目开展起来,让冰上项目强项更强,同时抓紧补上雪地项目短板,把没有开展的项目开展起来,促进冰雪运动在我国全面开展。
    }{
      \includegraphics[width = 2cm]{example-image-a}
    }
\end{vexample}

\begin{vexample}
    \textfigure[fig-pos = right-bottom]{
      “好风凭借力,送我上青云。”少年强则中国强,体育强则中国强。希望以北京冬奥会为契机,把冰雪运动各个项目开展起来,让冰上项目强项更强,同时抓紧补上雪地项目短板,把没有开展的项目开展起来,促进冰雪运动在我国全面开展。
    }{
      \includegraphics[width = 2cm]{example-image-a}
    }
\end{vexample}



\subsection{\cmd{left} 系列}

\begin{vexample}
    \textfigure[fig-pos = left]{
      “好风凭借力,送我上青云。”少年强则中国强,体育强则中国强。希望以北京冬奥会为契机,把冰雪运动各个项目开展起来,让冰上项目强项更强,同时抓紧补上雪地项目短板,把没有开展的项目开展起来,促进冰雪运动在我国全面开展。
    }{
      \includegraphics[width = 2cm]{example-image-a}
    }
\end{vexample}

\begin{vexample}
    \textfigure[fig-pos = left-top]{
      “好风凭借力,送我上青云。”少年强则中国强,体育强则中国强。希望以北京冬奥会为契机,把冰雪运动各个项目开展起来,让冰上项目强项更强,同时抓紧补上雪地项目短板,把没有开展的项目开展起来,促进冰雪运动在我国全面开展。
    }{
      \includegraphics[width = 2cm]{example-image-a}
    }
\end{vexample}

\begin{vexample}
    \textfigure[fig-pos = left-bottom]{
      “好风凭借力,送我上青云。”少年强则中国强,体育强则中国强。希望以北京冬奥会为契机,把冰雪运动各个项目开展起来,让冰上项目强项更强,同时抓紧补上雪地项目短板,把没有开展的项目开展起来,促进冰雪运动在我国全面开展。
    }{
      \includegraphics[width = 2cm]{example-image-a}
    }
\end{vexample}



\subsection{\cmd{top} 系列}

\begin{vexample}
    \textfigure[fig-pos = top-left]{
      “好风凭借力,送我上青云。”少年强则中国强,体育强则中国强。希望以北京冬奥会为契机,把冰雪运动各个项目开展起来,让冰上项目强项更强,同时抓紧补上雪地项目短板,把没有开展的项目开展起来,促进冰雪运动在我国全面开展。
    }{
      \includegraphics[width = 2cm]{example-image-a}
    }
\end{vexample}

\begin{vexample}
    \textfigure[fig-pos = top-right]{
      “好风凭借力,送我上青云。”少年强则中国强,体育强则中国强。希望以北京冬奥会为契机,把冰雪运动各个项目开展起来,让冰上项目强项更强,同时抓紧补上雪地项目短板,把没有开展的项目开展起来,促进冰雪运动在我国全面开展。
    }{
      \includegraphics[width = 2cm]{example-image-a}
    }
\end{vexample}




\subsection{\cmd{bottom} 系列}

\begin{vexample}
  \textfigure[fig-pos = bottom-left]{
    “好风凭借力,送我上青云。”少年强则中国强,体育强则中国强。希望以北京冬奥会为契机,把冰雪运动各个项目开展起来,让冰上项目强项更强,同时抓紧补上雪地项目短板,把没有开展的项目开展起来,促进冰雪运动在我国全面开展。
  }{
    \includegraphics[width = 2cm]{example-image-a}
  }
\end{vexample}

\begin{vexample}
  \textfigure[fig-pos = bottom-right]{
    “好风凭借力,送我上青云。”少年强则中国强,体育强则中国强。希望以北京冬奥会为契机,把冰雪运动各个项目开展起来,让冰上项目强项更强,同时抓紧补上雪地项目短板,把没有开展的项目开展起来,促进冰雪运动在我国全面开展。
  }{
    \includegraphics[width = 2cm]{example-image-a}
  }
\end{vexample}

\remark{
  \cmd{bottom-right} 也是 \cmd{fig-pos} 的默认值。
}


\subsection{ \cmd{fig-pos} 总结}

从上面的 \cmd{fig-pos} 示例可以看出,其实很容易记住这些方位,拿 \meta{pos1}-\meta{pos2}为例:
  \begin{itemize}
    \item \meta{pos1} 其实就是图片整体在文本的位置,就是上下左右四个位置
    \item \meta{pos2} 是位置的细节,比如 \meta{pos1} 为 \cmd{top}, \cmd{bottom} 的时候 \meta{pos2} 就是左右( \cmd{left}, \cmd{right} ),\meta{pos1} 为 \cmd{left}, \cmd{right} 的时候 \meta{pos2} 就是上中下(或顶部中部底部),即 \cmd{top}, \cmd{center}, \cmd{bottom}, 特别地 \cmd{center} 可以省略,即 \cmd{right} 为 \cmd{right-center} 的效果
  \end{itemize}



\subsection{ \cmd{text-ratio} 和 \cmd{text-width} }



\subsubsection{ \cmd{text-ratio} 的作用 }

\begin{vexample}
    \textfigure[fig-pos = right, text-ratio = 0.3]{
      “好风凭借力,送我上青云。”少年强则中国强,体育强则中国强。希望以北京冬奥会为契机,把冰雪运动各个项目开展起来,让冰上项目强项更强,同时抓紧补上雪地项目短板,把没有开展的项目开展起来,促进冰雪运动在我国全面开展。
    }{
      \includegraphics[width = 2cm]{example-image-a}
    }
\end{vexample}

\begin{vexample}
    \textfigure[fig-pos = right, text-ratio = 0.5]{
      “好风凭借力,送我上青云。”少年强则中国强,体育强则中国强。希望以北京冬奥会为契机,把冰雪运动各个项目开展起来,让冰上项目强项更强,同时抓紧补上雪地项目短板,把没有开展的项目开展起来,促进冰雪运动在我国全面开展。
    }{
      \includegraphics[width = 2cm]{example-image-a}
    }
\end{vexample}

\begin{vexample}
    \textfigure[fig-pos = right, text-ratio = 0.7]{
      “好风凭借力,送我上青云。”少年强则中国强,体育强则中国强。希望以北京冬奥会为契机,把冰雪运动各个项目开展起来,让冰上项目强项更强,同时抓紧补上雪地项目短板,把没有开展的项目开展起来,促进冰雪运动在我国全面开展。
    }{
      \includegraphics[width = 2cm]{example-image-a}
    }
\end{vexample}

\begin{vexample}
    \textfigure[fig-pos = right, text-ratio = 0.5, text-width = 8cm]{
      “好风凭借力,送我上青云。”少年强则中国强,体育强则中国强。希望以北京冬奥会为契机,把冰雪运动各个项目开展起来,让冰上项目强项更强,同时抓紧补上雪地项目短板,把没有开展的项目开展起来,促进冰雪运动在我国全面开展。
    }{
      \includegraphics[width = 2cm]{example-image-a}
    }
\end{vexample}


\subsubsection{ \cmd{text-width} 的作用 }

\begin{vexample}
    \textfigure[fig-pos = right, text-width = 6cm]{
      “好风凭借力,送我上青云。”少年强则中国强,体育强则中国强。希望以北京冬奥会为契机,把冰雪运动各个项目开展起来,让冰上项目强项更强,同时抓紧补上雪地项目短板,把没有开展的项目开展起来,促进冰雪运动在我国全面开展。
    }{
      \includegraphics[width = 2cm]{example-image-a}
    }
\end{vexample}

\begin{vexample}
    \textfigure[fig-pos = right, text-width = 10cm]{
      “好风凭借力,送我上青云。”少年强则中国强,体育强则中国强。希望以北京冬奥会为契机,把冰雪运动各个项目开展起来,让冰上项目强项更强,同时抓紧补上雪地项目短板,把没有开展的项目开展起来,促进冰雪运动在我国全面开展。
    }{
      \includegraphics[width = 2cm]{example-image-a}
    }
\end{vexample}



\subsubsection{ \cmd{text-ratio} 和 \cmd{text-width} 同时使用}

\begin{vexample}
    \textfigure[fig-pos = right, text-ratio = 0.5]{
      “好风凭借力,送我上青云。”少年强则中国强,体育强则中国强。希望以北京冬奥会为契机,把冰雪运动各个项目开展起来,让冰上项目强项更强,同时抓紧补上雪地项目短板,把没有开展的项目开展起来,促进冰雪运动在我国全面开展。
    }{
      \includegraphics[width = 2cm]{example-image-a}
    }
\end{vexample}

\begin{vexample}
    \textfigure[fig-pos = right, text-ratio = 0.5, text-width = 8cm]{
      “好风凭借力,送我上青云。”少年强则中国强,体育强则中国强。希望以北京冬奥会为契机,把冰雪运动各个项目开展起来,让冰上项目强项更强,同时抓紧补上雪地项目短板,把没有开展的项目开展起来,促进冰雪运动在我国全面开展。
    }{
      \includegraphics[width = 2cm]{example-image-a}
    }
\end{vexample}



\section{试卷排版示例}


\subsection{选择题}

\begin{enumerate}
  \item \textfigure[fig-pos = bottom]{
    如图, 网格纸上小正方形的边长为 $1$ , 粗实线画出的是某几何体的三视图, 该几何体由一平面将一圆柱截去一部分后所得, 则该几何体的体积为 ( )
  }{
    \includegraphics[width = 3cm]{example-image-a}
  }
    \choices{
      $90\pi$ &&
      $63\pi$ &&
      $42\pi$ &&
      $36\pi$ &&
    }
\end{enumerate}


\subsection{填空题}

\begin{enumerate}[start = 11]
  \item \textfigure[fig-pos = right-top]{
    执行如图的程序框图, 如果输入的 $a = -1$, 则输出的 $S = \underline{\qquad}$
  }{
    \includegraphics[height = 3cm]{example-image-a}
  }
\end{enumerate}



\subsection{解答题}

\begin{enumerate}[start = 18]
  \item 
    如图, 在平面直角坐标系 $x O y$ 中, 椭圆 $C$ 过点 $\left(\sqrt{3}, \frac{1}{2}\right)$, 焦点为 $F_{1}(-\sqrt{3}, 0), F_{2}(\sqrt{3}, 0)$, 圆 $O$ 的直径为 $F_{1} F_{2}$.
    \begin{enumerate}
      \item 求椭圆 $C$ 及圆 $O$ 的方程;
      \item 设直线 $l$ 与圆 $O$ 相切于第一象限内的点 $P$.
      \begin{enumerate}[]
          \item 若直线 $l$ 与椭圆 $C$ 有且只有一个公共点, 求点 $P$ 的坐标;
          \item
            直线 $l$ 与椭圆 $C$ 交于 $A, B$ 两点. 若 $\triangle O A B$ 的面积为 $\frac{2 \sqrt{6}}{7}$, 求直 线 $l$ 的方程.   
        \end{enumerate}
    \end{enumerate}
    
  \item \textfigure{
    如图, 在正三棱柱 $A B C-A_{1} B_{1} C_{1}$ 中, $A B=A A_{1}=2$, 点 $P, Q$ 分别为 $A_{1} B_{1}, B C$ 的中点.
    如图, 在正三棱柱 $A B C-A_{1} B_{1} C_{1}$ 中, $A B=A A_{1}=2$, 点 $P, Q$ 分别为 $A_{1} B_{1}, B C$ 的中点.
    % 如图, 在正三棱柱 $A B C-A_{1} B_{1} C_{1}$ 中, $A B=A A_{1}=2$, 点 $P, Q$ 分别为 $A_{1} B_{1}, B C$ 的中点.
      \begin{itemize}
        \item 求异面直线 $B P$ 与 $A C_{1}$ 所成角的余弦值;
        \item 求直线 $C C_{1}$ 与平面 $A Q C_{1}$ 所成角的正弦值.
      \end{itemize}
    }{
      \includegraphics[width = 5cm]{example-image-a}
    }
\end{enumerate}

\end{document}
